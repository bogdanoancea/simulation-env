\documentclass{article}
\usepackage{hyperref}
\usepackage{xcolor}
\usepackage{geometry}
\usepackage{fancyhdr}
\usepackage{tocloft}
\usepackage{titlesec}
\usepackage{amsmath}
\usepackage{verbatim}

% Page setup
\geometry{
	a4paper,
	left=25mm,
	right=25mm,
	top=25mm,
	bottom=25mm,
}

% Header and Footer
\pagestyle{fancy}
\fancyhf{}
\lhead{Simulating Mobile Phone Network Data in Bucharest}
\rhead{\thepage}
\cfoot{}

% Table of Contents formatting
\renewcommand{\cftsecleader}{\cftdotfill{\cftdotsep}}

% Hyperlink setup
\hypersetup{
	colorlinks=true,
	linkcolor=blue,
	urlcolor=blue,
}

\begin{document}
	
	\title{Simulating Mobile Phone Network Data in Bucharest Using VEINS and INET Frameworks}
	\author{}
	\date{}
	
	\maketitle
	
	\tableofcontents
	\newpage
	
	\section{Overview}
	
	This guide provides comprehensive instructions and code examples to simulate mobile phone network data with geolocation of devices on the Bucharest map over 24 hours for \textbf{100,000 SIM cards} using the \textbf{VEINS} and \textbf{INET} frameworks. Each phone can have between \textbf{1 and 3 SIM cards}, and mobility patterns are based on Bucharest's points of interest (POIs) such as schools, airports, train stations, and bus stations. The simulation includes two modes of transportation: \textbf{pedestrian} and \textbf{vehicle}. Additionally, the network configuration incorporates antenna locations and types from \textbf{OpenCellID} data for Bucharest.
	
	\section{Prerequisites}
	
	Ensure you have the following software and data installed and prepared:
	
	\begin{itemize}
		\item \textbf{Software:}
		\begin{itemize}
			\item \textbf{OMNeT++} (version 5.6 or later)
			\item \textbf{INET Framework} (version 4.2 or later)
			\item \textbf{VEINS} (version 5.1 or later)
			\item \textbf{SUMO} (version 1.7.0 or later)
			\item \textbf{Python} (for processing scripts)
		\end{itemize}
		\item \textbf{Data:}
		\begin{itemize}
			\item \textbf{OpenStreetMap (OSM) Data} for Bucharest
			\item \textbf{OpenCellID Data} for Bucharest
		\end{itemize}
	\end{itemize}
	
	\section{Data Preparation}
	
	\subsection{Bucharest Map Data}
	
	\begin{enumerate}
		\item \textbf{Download OSM Data:}
		\begin{itemize}
			\item Download Bucharest map data from OpenStreetMap in \texttt{.osm} format.
			\item \href{https://www.openstreetmap.org/export#map=12/44.4268/26.1025}{OpenStreetMap Export}
		\end{itemize}
		\item \textbf{Convert OSM Data to SUMO Network:}
	\end{enumerate}
	
	\begin{verbatim}
		netconvert --osm-files bucharest.osm -o bucharest.net.xml
	\end{verbatim}
	
	\subsection{OpenCellID Antenna Data}
	
	\begin{enumerate}
		\item \textbf{Download OpenCellID Data:}
		\begin{itemize}
			\item Register at \href{https://opencellid.org/}{OpenCellID} to access cell tower data.
			\item Download data for Bucharest in CSV format.
		\end{itemize}
		\item \textbf{Process OpenCellID Data:}
		\begin{itemize}
			\item Filter data to include antennas located in Bucharest.
			\item Extract necessary fields: Cell ID, Latitude, Longitude, and Antenna Type.
		\end{itemize}
	\end{enumerate}
	
	\section{Setting Up the Simulation Environment}
	%TODO : the easy way: download the virtual machine with all software packages installed
	%TODO: the norm way .... install each component.
	\subsection{Installing OMNeT++, INET, and VEINS}
	
	\begin{enumerate}
		\item \textbf{Install OMNeT++:}
		\begin{itemize}
			\item Follow the \href{https://doc.omnetpp.org/omnetpp/InstallGuide.pdf}{OMNeT++ Installation Guide}.
		\end{itemize}
		\item \textbf{Install INET Framework:}
		\begin{itemize}
			\item Clone the INET repository into the OMNeT++ workspace.
			\begin{verbatim}
				cd omnetpp-5.x/samples
				git clone https://github.com/inet-framework/inet.git
			\end{verbatim}
			\item Import INET into the OMNeT++ IDE and build it.
		\end{itemize}
		\item \textbf{Install VEINS:}
		\begin{itemize}
			\item Clone the VEINS repository into the OMNeT++ workspace.
			\begin{verbatim}
				git clone https://github.com/sommer/veins.git
			\end{verbatim}
			\item Import VEINS into the OMNeT++ IDE and build it.
		\end{itemize}
	\end{enumerate}
	
	\subsection{Integrating SUMO with VEINS}
	
	\begin{enumerate}
		\item \textbf{Configure SUMO:}
		\begin{itemize}
			\item Ensure SUMO is installed and accessible in your system's PATH.
			\item Set the environment variable \texttt{SUMO\_HOME} to point to the SUMO installation directory.
		\end{itemize}
		\item \textbf{Test Integration:}
		\begin{itemize}
			\item Run a VEINS example simulation to verify integration with SUMO.
		\end{itemize}
	\end{enumerate}
	
	\section{Defining Mobility Patterns with SUMO}
	
	\subsection{Defining Points of Interest (POIs)}
	
	Create a file named \texttt{bucharest.poi.xml} with the following content:
	
	\begin{verbatim}
		<additional>
		<!-- Henri Coanda International Airport -->
		<poi id="airport" x="26.1025" y="44.5720" />
		<!-- Gara de Nord Train Station -->
		<poi id="train_station" x="26.0840" y="44.4440" />
		<!-- Piata Unirii Bus Station -->
		<poi id="bus_station" x="26.1030" y="44.4270" />
		<!-- Add more POIs such as schools, malls, parks -->
		</additional>
	\end{verbatim}
	
	\subsection{Generating Mobility Routes}
	
	\subsubsection{Generating Random Trips}
	
	Use SUMO's \texttt{randomTrips.py} to generate trips for pedestrians and vehicles.
	
	\textbf{Pedestrian Trips:}
	
	\begin{verbatim}
		python $SUMO_HOME/tools/randomTrips.py \
		-n bucharest.net.xml \
		-o pedestrian.trips.xml \
		--additional-files bucharest.poi.xml \
		--pedestrians \
		--trip-attributes="type=\"pedestrian\"" \
		-e 86400 \
		-p 0.1 \
		--seed 42 \
		--validate
	\end{verbatim}
	
	\textbf{Vehicle Trips:}
	
	\begin{verbatim}
		python $SUMO_HOME/tools/randomTrips.py \
		-n bucharest.net.xml \
		-o vehicle.trips.xml \
		--additional-files bucharest.poi.xml \
		--trip-attributes="type=\"vehicle\"" \
		-e 86400 \
		-p 0.05 \
		--seed 42 \
		--validate
	\end{verbatim}
	
	\subsubsection{Converting Trips to Routes}
	
	\begin{verbatim}
		duarouter \
		-n bucharest.net.xml \
		-t pedestrian.trips.xml \
		-o pedestrian.rou.xml \
		--additional-files bucharest.poi.xml
		
		duarouter \
		-n bucharest.net.xml \
		-t vehicle.trips.xml \
		-o vehicle.rou.xml \
		--additional-files bucharest.poi.xml
	\end{verbatim}
	
	\subsubsection{Creating the SUMO Configuration File}
	
	Create a file named \texttt{bucharest.sumo.cfg}:
	
	\begin{verbatim}
		<configuration>
		<input>
		<net-file value="bucharest.net.xml"/>
		<route-files value="pedestrian.rou.xml,vehicle.rou.xml"/>
		<additional-files value="bucharest.poi.xml"/>
		</input>
		<time>
		<begin value="0"/>
		<end value="86400"/>
		</time>
		</configuration>
	\end{verbatim}
	
	\section{Configuring Network Nodes and SIM Cards}
	
	\subsection{Creating the Network Topology (NED Files)}
	
	\subsubsection{Mobile Network NED File (\texttt{MobileNetwork.ned})}
	
	\begin{verbatim}
		package mobile.simulation;
		
		import inet.common.misc.ThruputMeteringChannel;
		import inet.node.inet.AdhocHost;
		
		network MobileNetwork
		{
			parameters:
			int numDevices = default(50000); // Each device can have multiple SIM cards
			submodules:
			// Cellular network module
			cellularNetwork: CellularNetwork {
				@display("p=100,100");
			}
			// Mobile devices
			devices[numDevices]: MobileDevice {
				@display("p=200,100");
			}
		}
	\end{verbatim}
	
	\subsubsection{Channel Definition}
	
	\begin{verbatim}
		channel mediumLine extends ThruputMeteringChannel
		{
			delay = 0us;
			datarate = 1Gbps;
		}
	\end{verbatim}
	
	\subsection{Implementing the Mobile Device Module}
	
	\subsubsection{Mobile Device NED File (\texttt{MobileDevice.ned})}
	
	\begin{verbatim}
		package mobile.simulation;
		
		import inet.node.inet.AdhocHost;
		import inet.mobility.contract.IMobility;
		import veins.base.modules.BaseApplLayer;
		
		module MobileDevice extends AdhocHost
		{
			parameters:
			int numSims;
			string modeOfTransport; // "pedestrian" or "vehicle"
			@display("i=device/smartphone");
			submodules:
			app: BaseApplLayer {
				@display("p=100,100");
			}
			mobility: <default("TraCIMobility")> like IMobility if hasPar("mobilityType");
		}
	\end{verbatim}
	
	\subsubsection{Mobile Device Implementation (\texttt{MobileDevice.cc})}
	
	Create \texttt{MobileDevice.cc} in the \texttt{src} directory:
	
	\begin{verbatim}
		#include <omnetpp.h>
		#include "MobileDevice.h"
		#include "inet/mobility/traci/TraCIMobility.h"
		#include "inet/common/ModuleAccess.h"
		
		Define_Module(MobileDevice);
		
		void MobileDevice::initialize(int stage)
		{
			AdhocHost::initialize(stage);
			
			if (stage == inet::INITSTAGE_LOCAL) {
				// Assign SIM cards
				int numSims = par("numSims").intValue();
				for (int i = 0; i < numSims; ++i) {
					std::string simId = "SIM" + std::to_string(getIndex() * 3 + i);
					simCards.push_back(simId);
				}
				
				// Set mode of transport
				modeOfTransport = par("modeOfTransport").stdstringValue();
				
				// Mobility setup
				mobility = inet::getModuleFromPar<inet::TraCIMobility>(par("mobilityModule"), this);
			}
		}
	\end{verbatim}
	
	\subsubsection{Mobile Device Header (\texttt{MobileDevice.h})}
	
	Create \texttt{MobileDevice.h} in the \texttt{src} directory:
	
	\begin{verbatim}
		#ifndef MOBILEDEVICE_H
		#define MOBILEDEVICE_H
		
		#include "inet/node/inet/AdhocHost.h"
		#include "inet/mobility/traci/TraCIMobility.h"
		
		class MobileDevice : public inet::AdhocHost
		{
			protected:
			std::vector<std::string> simCards;
			std::string modeOfTransport;
			inet::TraCIMobility* mobility;
			
			virtual void initialize(int stage) override;
		};
		
		#endif // MOBILEDEVICE_H
	\end{verbatim}
	
	\section{Configuring the Cellular Network with OpenCellID Data}
	
	\subsection{Processing OpenCellID Data}
	
	\begin{enumerate}
		\item \textbf{Filter Antennas for Bucharest:}
		
		Use a Python script to filter antennas within Bucharest's bounding box.
		
		\begin{verbatim}
			import pandas as pd
			
			# Load OpenCellID data
			data = pd.read_csv('opencellid_bucharest.csv')
			
			# Define bounding box for Bucharest
			min_lat, max_lat = 44.3, 44.6
			min_lon, max_lon = 25.9, 26.3
			
			# Filter data
			bucharest_antennas = data[
			(data['lat'] >= min_lat) & (data['lat'] <= max_lat) &
			(data['lon'] >= min_lon) & (data['lon'] <= max_lon)
			]
			
			# Save filtered data
			bucharest_antennas.to_csv('bucharest_antennas.csv', index=False)
		\end{verbatim}
		
		\item \textbf{Generate Antenna NED Definitions:}
		
		\begin{verbatim}
			antennas = bucharest_antennas.to_dict('records')
			
			with open('AntennaNetwork.ned', 'w') as f:
			f.write('package mobile.simulation;\n\n')
			f.write('import inet.node.inet.AccessPoint;\n\n')
			f.write('network AntennaNetwork {\n')
				f.write('    submodules:\n')
				for idx, antenna in enumerate(antennas):
				x_pos = antenna["lon"] * 100000  # Scale coordinates appropriately
				y_pos = antenna["lat"] * 100000
				f.write(f'        antenna{idx}: Antenna {{\n')
						f.write(f'            @display("p={x_pos},{y_pos}");\n')
						f.write(f'            antennaType = "{antenna["antenna_type"]}";\n')
						f.write('        }\n')
					f.write('}\n')
			\end{verbatim}
			
			\textbf{Note:} Multiply coordinates to match the simulation's scale.
			
		\end{enumerate}
		
		\subsection{Defining Base Stations in the Simulation}
		
		Include the antenna network in your main network.
		
		\textbf{Update \texttt{MobileNetwork.ned}:}
		
		\begin{verbatim}
			network MobileNetwork
			{
				parameters:
				int numDevices = default(50000);
				submodules:
				antennaNetwork: AntennaNetwork {
					@display("p=50,50");
				}
				devices[numDevices]: MobileDevice {
					@display("p=200,100");
				}
			}
		\end{verbatim}
		
		\textbf{Define Antenna Module (\texttt{Antenna.ned}):}
		
		\begin{verbatim}
			package mobile.simulation;
			
			import inet.node.inet.AccessPoint;
			
			module Antenna extends AccessPoint
			{
				parameters:
				string antennaType;
				@display("i=block/celltower");
			}
		\end{verbatim}
		
		\section{Setting Up the Simulation Configuration (\texttt{omnetpp.ini})}
		
		Create an \texttt{omnetpp.ini} file with the following content:
		
		\begin{verbatim}
			[General]
			network = MobileNetwork
			sim-time-limit = 24h
			
			# Number of devices (each device has between 1 and 3 SIM cards)
			**.numDevices = 50000
			
			# Mobility Configuration
			**.devices[*].mobilityModule = "^.^.mobility"
			**.devices[*].mobility.typename = "TraCIMobility"
			**.devices[*].mobility.externalId = format("device%d", index)
			**.devices[*].mobility.host = this
			
			# Assign number of SIM cards per device
			**.devices[*].numSims = intuniform(1, 3)
			
			# Mode of Transport Assignment
			**.devices[*].modeOfTransport = expr(index % 2 == 0 ? "pedestrian" : "vehicle")
			
			# Wireless Interface Configuration
			**.devices[*].wlan[0].typename = "WirelessInterface"
			**.devices[*].wlan[0].mac.typename = "Ieee80211Mac"
			**.devices[*].wlan[0].radio.typename = "Ieee80211Radio"
			
			# Antenna Configuration
			**.antennaNetwork.antenna*.wlan[0].typename = "WirelessInterface"
			**.antennaNetwork.antenna*.wlan[0].mac.typename = "Ieee80211Mac"
			**.antennaNetwork.antenna*.wlan[0].radio.typename = "Ieee80211Radio"
			
			# Configure Antenna Types
			for idx in range(numAntennas):
			antennaType = antennas[idx]["antennaType"]
			**.antennaNetwork.antenna${idx}.antennaType = "${antennaType}"
			
			# VEINS and SUMO Configuration
			*.manager.typename = "TraCIScenarioManager"
			*.manager.configFile = "bucharest.sumo.cfg"
			*.manager.launchConfig = "-c $configFile"
			*.manager.port = 9999
			*.manager.autoShutdown = true
			*.manager.client.launcher = "sumo-launchd.py"
			
			# Logging Configuration
			**.scalar-recording = true
			**.vector-recording = true
		\end{verbatim}
		
		\section{Running the Simulation and Collecting Data}
		
		\subsection{Starting SUMO}
		
		Start SUMO to communicate with VEINS:
		
		\begin{verbatim}
			sumo-gui -c bucharest.sumo.cfg --remote-port 9999
		\end{verbatim}
		
		Alternatively, allow VEINS to start SUMO automatically.
		
		\subsection{Running the Simulation}
		
		\begin{enumerate}
			\item \textbf{Build the Project:}
			\begin{itemize}
				\item In the OMNeT++ IDE, build the project to compile all modules.
			\end{itemize}
			\item \textbf{Execute the Simulation:}
			\begin{itemize}
				\item Run the simulation using \texttt{omnetpp.ini}.
				\item Monitor for any errors or warnings.
			\end{itemize}
		\end{enumerate}
		
		\subsection{Collecting Geolocation Data}
		
		Implement data logging in \texttt{MobileDevice.cc}:
		
		\begin{verbatim}
			void MobileDevice::handleMessage(cMessage *msg)
			{
				// Log current position
				inet::Coord position = mobility->getCurrentPosition();
				simtime_t timestamp = simTime();
				
				// Record position vector for each SIM card
				for (auto &simId : simCards) {
					std::string vectorName = "SIM_" + simId + "_Position";
					cOutVector *positionVector = new cOutVector(vectorName.c_str());
					positionVector->recordWithTimestamp(timestamp, position.distance(inet::Coord::ZERO));
				}
				
				// Continue with message handling
				AdhocHost::handleMessage(msg);
			}
		\end{verbatim}
		
		\textbf{Note:} The data will be recorded in the simulation output files. Use OMNeT++ analysis tools or export the data for further processing.
		
		\section{Conclusion}
		
		This guide provides the full code and steps required to set up a simulation of mobile phone network data with geolocation of devices on the Bucharest map over 24 hours for \textbf{100,000 SIM cards} using \textbf{VEINS} and \textbf{INET} frameworks. By incorporating real antenna data from \textbf{OpenCellID} and modeling realistic mobility patterns with \textbf{SUMO}, the simulation offers a detailed representation of network and mobility behaviors.
		
		\subsection*{Next Steps}
		
		\begin{itemize}
			\item \textbf{Performance Optimization:}
			\begin{itemize}
				\item Simulating 100,000 devices is resource-intensive. Consider using high-performance computing resources or reducing the number of devices for testing.
			\end{itemize}
			\item \textbf{Data Analysis:}
			\begin{itemize}
				\item After the simulation, analyze the collected geolocation data using appropriate data analysis tools.
			\end{itemize}
			\item \textbf{Enhancements:}
			\begin{itemize}
				\item Incorporate detailed mobility models, advanced network protocols, or environmental factors to enhance realism.
			\end{itemize}
		\end{itemize}
		
		\section{References}
		
		\begin{itemize}
			\item \textbf{OMNeT++ Documentation:} \href{https://doc.omnetpp.org/omnetpp/manual/}{OMNeT++ Manual}
			\item \textbf{INET Framework:} \href{https://inet.omnetpp.org/}{INET Framework}
			\item \textbf{VEINS Framework:} \href{https://veins.car2x.org/}{VEINS}
			\item \textbf{SUMO Documentation:} \href{https://sumo.dlr.de/docs/index.html}{SUMO Wiki}
			\item \textbf{OpenCellID:} \href{https://opencellid.org/}{OpenCellID}
			\item \textbf{OpenStreetMap:} \href{https://www.openstreetmap.org/}{OpenStreetMap}
		\end{itemize}
		
		\textbf{Note:} The provided code and configurations serve as a starting point. Further adjustments and optimizations may be necessary to run the simulation successfully in your environment.
		
	\end{document}
